\usepackage[latin1]{inputenc} %acentos
\usepackage[spanish,es-tabla]{babel} %acentos
\usepackage{amsmath} %Matematicas
\usepackage{amsfonts} %Matematicas
\usepackage{amssymb} %Matematicas
\usepackage{graphicx} %imagens
\usepackage[left=3cm,right=2.5cm,top=3cm,bottom=2.5cm]{geometry} %margenes
\usepackage{multicol} %multiple columna
\usepackage[usenames]{color} %texto de color
\usepackage{float} % posicion de las iguras
\usepackage[hidelinks]{hyperref} %referencia de un link
\usepackage{bm} %Matematica negrita
\usepackage{subfigure} % subfiguras 
\usepackage{anyfontsize} % configura el tama�o de la letra personalizada
\usepackage{setspace}%espaciado
\usepackage{enumerate}%congura listas 1. a. A. I.
%\usepackage[format=plain,labelfont={bf,it},textfont=it]{caption}% figura en NEGRITA Y CURSIVA
%\usepackage[format=plain,font=it]{caption}%CAPTION en SOLO CURSIVA
\usepackage[mathscr]{eucal} %letras goticas con $\mathscr$

%%%%%%%%% Encabezado y pie de pagina
\usepackage{fancybox}% caja 3D \shadowbox{argumento}
%\usepackage{fancyhdr}
%\pagestyle{fancy}
%\fancyhf{} %limpia el encabezado
%\fancyhead[R]{\includegraphics[scale=0.06]{escudo}} % agrega encabezado
%\fancyhead[L]{{\large \textit{}}} % agrega encabezado
%\fancyfoot[L]{\textit{Ingenier�a Electr�nica}} %agrega pie de pagina
%\fancyfoot[C]{\thepage} %agrega pie de pagina
%\fancyfoot[R]{\textit{Materia}} %agrega pie de pagina
%\renewcommand{\headrulewidth}{2pt}
%\renewcommand{\footrulewidth}{2pt}
%\footskip=0pt

%%%%%%%%%%%%% configuracion de secciones
%\usepackage{sectsty} % centrar secciones
%\sectionfont{\centering} % centrar secciones 

%%%%%%%%% tablas 
\usepackage{array} % permite usar el ancho de columna m{}
\usepackage{longtable} % tablas largas 
\newcolumntype{E}{>{$}c<{$}} % tablas en modo ecuacion

%%%%%%%% angulo de los fasores
\usepackage{steinmetz} %\phase

\parindent=0mm % tama�o de sangria 
\renewcommand{\labelitemi}{$\bullet$} % cambia el estilo de las vi�etas
%\pagestyle{empty} % no enumera las paginas
\graphicspath{{Imagenes/}}
\columnsep=4.22mm %separaci�n de las columnas
\renewcommand{\arraystretch}{1.2}% Espacio entre filas de las tablas 
%\renewcommand{\contentsname}{CONTENIDO}%tabla de contenido (va dentro del documento)

\decimalpoint

% Etiqueta de figura y tabla a la izquierda seguida de punto.
\usepackage[justification=centering, singlelinecheck=false, labelsep = period]{caption}

\addtocontents{toc}{~\hfill\textbf{p�g.}\par}
\usepackage{cancel}

\usepackage{lscape}% ROTAR LA PAGINA
\usepackage{listings} %escribir pseudocodigo


\renewcommand{\thesection}{\arabic{section}} %secciones en romano
\renewcommand{\thesubsection}{\arabic{section}.\arabic{subsection}} %subsecciones en arabico

\usepackage{helvet}
\renewcommand{\familydefault}{\sfdefault} % Letra Arial
\thispagestyle{empty}
\usepackage{multirow}
